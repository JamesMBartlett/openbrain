%!TEX root = main.tex
\section{Asynchronous Signal Error Backpropagation}

In this section we introduce an asynchronous signal error backpropagation method,
as a general framework for updating parameters using the policy gradient. In traditional
supervised learning, neural networks propagate an error gradient to almost every neuron on every layer.
This method is adventageous since it it is easily optimized using GPUs and tensor multiplication, but measures must be 
taken to introduce sparsity and prevent overfitting such as dropout and ReLU activations. 

Our framework immediately invokes sparsity of both neuronal connections and neuronal firings and therefore would perform suboptimally 
by such methods. Fortunately our framework can take advantage of chain rule as a graph transition process. Consider the following situation.

\begin{definition}
	Select subsets of neurons $N_I, N_O$ to denote the i\textbf{nput and output neuron}s for an openbrain $O.$ Then given a dataset $D$ of datapairs $(x,y)$ where $x \in \mathbb{R}^{|N_I|}, y \in \mathbb{R}^{|N_O|}$, we define the \textbf{error or loss} of $O$ on $D$ as 
	\begin{equation}
		\begin{aligned}
			V(n,t) &\leftarrow x, &n \in N_I \\
			E(t+t_e) &= \frac{1}{2} \|V(m,t+t_e) - y\|^2, &m \in N_O
		\end{aligned}
	\end{equation}
	where $(x,y) \in D$ and $[t,t+t_e]$ is the \textbf{evaluation interval}.
\end{definition}

Learning is then defined as the optimization of the error function over the evaluation interval with respect to the connection parameters which lead to the final behaviour. One might optimize learning so that the most optimal behaviour minimizes the evaluation interval by letting $t_e \to 0$ as $t \to \infty,$ but we hold $t_e$ constant in the initial formulation.


Suppose we have a neuron $n$ as in \todo{Insert neuron figure here} that has just been activated at $t = t_0.$ This leads to a series of activations tagged with the activation of $n$ at $t_0$ 

