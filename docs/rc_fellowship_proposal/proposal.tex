%%%%%%%%%%%%%%%%%%%%%%%%%%%%%%%%%%%%%%%%%%%%%%%%%%%%%%%%%%%%%%%%%%
%%%                      Homework _                            %%%
%%%%%%%%%%%%%%%%%%%%%%%%%%%%%%%%%%%%%%%%%%%%%%%%%%%%%%%%%%%%%%%%%%

\documentclass[letter]{article}

\usepackage{lipsum}
\usepackage[pdftex]{graphicx}
\usepackage[margin=1.5in]{geometry}
\usepackage[english]{babel}
\usepackage{listings}
\usepackage{amsthm}
\usepackage{amssymb}
\usepackage{framed} 
\usepackage{amsmath}
\usepackage{titling}
\usepackage{fancyhdr}

\pagestyle{fancy}


\newtheorem{theorem}{Theorem}
\newtheorem{definition}{Definition}

\newenvironment{menumerate}{%
  \edef\backupindent{\the\parindent}%
  \enumerate%
  \setlength{\parindent}{\backupindent}%
}{\endenumerate}







%%%%%%%%%%%%%%%
%% DOC INFO %%%
%%%%%%%%%%%%%%%

\title{RC Research Fellowship Proposal: OpenBrain}
\author{William Guss\\26793499\\wguss@berkeley.edu}

\fancyhead[L]{RC Research Fellowship}
\fancyhead[CO]{Proposal}
\fancyhead[CE]{GUSS}
\fancyhead[R]{\thepage}
\fancyfoot[LR]{}
\fancyfoot[C]{}
\usepackage{csquotes}

%%%%%%%%%%%%%%

\begin{document}
\maketitle
\thispagestyle{empty}


%%%%%%% Be sure to set the counter and use menumerate
\section{Background} 
\section{Goals}
There are essentially four goals of the OpenBrain project. 
\begin{itemize}
\item Build a massiveley parallel Beowulf cluster of parallela computers controlled using MPI on ArchLinux. 
\item Create an always online, turing complete modification to the recursive neural network algorithm whose fitness
is determined by the Universal Intelligence Measure described in (Legg and Veness, 2011). The algorithm must have the 
following constraints:
    \begin{itemize}
        \item In the spirit of John Conway's turing complete Game of Life, the individual neural nodes must follow
         arbitrarily simple rules in a decentralized fashion.
        \item \emph{Training} is unsupervised and occurs over the lifetime of an \emph{instance} of the open brain,
        such that the aforementioned governing rules are modified with respect to the fitness of the instance.
    \end{itemize}
\item Implement each neural node as an Erlang process distributed across the Beowulf cluster asynchronously.
\item Provide always on input/output to the OpenBrain cluster in similar fashion to that done in Google DeepMind's
  Deep Reinforcement Learning.
\end{itemize} 


\section{Plans}
The OpenBrain project will commense in two phases. First the algorithm in question must be exactly theorized and mathematical
guarentees about its capabilities must be given. Then given that motivation, the project must be implemented in hardware and software.

For the theory behind OpenBrain, we apply the nuronal model proposed in (McCulloch and Pitts, 1949) and assume that each neuron
functions according Conwaynian turing complete rules; that is in particular, we apply the synaptogenesis model suggested in
(Thomas et al, 2015) with parameters dictated by a universal intelligence indicator function.

For any neuron $\pi$ in the OpenBrain algorithm, a list of axonally connected posterior neurons $P_\pi$ is stored along with
a list of proximal neighboring neurons, $N_\pi.$ The neuron $\pi$ is efficiently implimented as asyncrhonous thread with a message
queue such that neurons connected to its anterior can \ activate and provide a voltage on $\pi$ such that $\pi$ itself
will activate. Furthermore, each neuron has the capacity to signal to a particular proximal neuron in order to form a new dendritic or axonal 
synapse. 

Under this lightweight model, we will experiment freely with different schemes of synaptogenesis and learning. Upon finding the most
appropriate model, guarentees will be made on its computational capacity of the algorithm, and with any hope the second
phase of the project will commense.

The second phase of the project involves implementing the algorithm on a massiveley parallel system, and hence the initial
focus of this phase is the construction of a 32 node Beowulf cluster of parallela computers. The necessity for this construction
is due to the large amount of CPU hours required to have the algorithm run continually. It would not be cost effective to run
each OpenBrain instance on per CPU hour clusters such as Berkeley's own PSI cluster.  Furthermore control of the hardware
provides the opportunity to maximize the performance of each instance with respect to parallel processing power.

Once completed the cluster will be linked directly to a build system using hooks from Github which automatically pushes Erlang
OTP code for the OpenBrain algorithm to every node. This will allow for easy prototyping and live code hot swapping. 

Finally much research can be done in a variety of external environments using the universal intelligence indicator and other metrics that evaluate performance.
Specifically, OpenBrain will be compared to Deep Reinforcement Learning. These applications and comparisons will be 
understood with more specificity as the project matures.

\section{Qualifications}
\textbf{Coursework}
\begin{menumerate}
\item Honors Real Analysis (MATH H104) [A] - Useful tools for providing mathematical guarentees on computability.
\item Data Structures and Algorithms (CS61B) [A] - Useful for ensuring optimal computational complexity as it pertains to the project.
\end{menumerate}


\textbf{Past Research} \\
Over the past year, I've been developing an algorithmic generalization of McCulloch and Pitts feed-forward neural networks
called functional neural networks. The project has provided me the education and the toolset to give actuall guarentees on 
a variety of different neural network algorithms. The project received a prize from the AAAI at the Intel International Science and Engineering Fair.
Attached is a copy of the paper, currently in progress, describing the project. Hopefully that may provide a sample of the quality
of research I will do.

\textbf{Miscelaneous} \\
Over the past semester I've attended Peter Bartlett's Statistical Machine Learning reading group weekly.


\section{Budget}
The majority of the budgett of this project deals directly with the construction of the Beowulf cluster.


\vspace{7mm}
\centering
\begin{tabular}{ |p{2cm}|p{6cm}|p{1.5cm}|p{1cm}| p{1.5cm}|
 }
 \hline
 \multicolumn{5}{|c|}{Budget} \\
 \hline
Item& Use& Quantity&Price & Total\\
 \hline
  Parallela & Used as the primary computation unit for the cluster.  Has 18 discrete cores and low power
  consumption & 32 & \$100  & \$3200\\

 \hline
\end{tabular}
\end{document}