%!TEX root = ../main.tex
\subsection{Continuous Time Universal Intelligence Measure}

In alignment with the philosophy which predicates OpenBrain, we wish to extend the evaluation of our algorithm well beyond its performance in supervised learning tasks; that is, what can be said about the intelligence of OpenBrain as an agent in an environment? A metric more condusive to implementing general intelligence is needed. The answer will motivate an important discussion of representation theory annd learning rules.
\subsection{Universal Intelligence}
A well established\cite{legg2008machine, mnih2015human, rathmanner2011philosophical}  machine intelligence measure is the Universal Intelligence Measure proposed by Legg and Hutter \cite{legg2007universal}. Drawing from a large amount of disparate literature on the subject they develop a consise definition of the intelligence of an \emph{agent} in an \emph{environment}. 
\todo[inline]{The following exposition may be unnecessary, and we should consider just stating the result of Legg and Hutter.}

Environment-agent interaction is defined with respect to an observation space, $\mathcal{O}$, and an action space, $\mathcal{A}$, both of which consist of abstract symbols. The perception space, $\mathcal{P} \subset \mathcal{O} \times \mathbb{R}$, is the combination of observations and rewards.
\begin{definition}
	An \textbf{environment} $\mu$ is a probability measure, specifically defining
	\begin{equation}
			\mu(o_kr_k | o_1r_1a_1 \dots o_{k-1}r_{k-1}a_{k-1}),
	\end{equation}
	the probability of observing $o_kr_k \in \mathcal{P}$ given a history, a string $o_1\dots a_{k-1} \in \bigtimes_{i=1}^k \mathcal{P} \times \mathcal{A}.$
\end{definition}

In the same light the agent definition is given.
\begin{definition}
	An \textbf{agent} $\pi$ is a probability measure, giving
	\begin{equation}
		\pi(a_k | o_1r_1a_1 \dots a_{k-1}o_kr_k)
	\end{equation}
	the probability of the possible action $a_k$ being enacted by $\pi$ being the environment-agent interaction history.
\end{definition}
\todo[inline]{Include agent environment interaction picture? Would this section be clear given that the reader has explored \cite{legg2007universal}?}

Having defined the basic framework, \cite{legg2007universal} gives a definition for Universal Intelligence. Let $E$ be the space of all turing complete reward environments, and $K: E \to  \mathbb{R}$ be the Kolmogorov complexity of an environment. This complexity is calculated with respect to the length of the string with which a reference machine $\mathcal{U}$ generates the environment.
\begin{definition} \label{uim}
	If $\pi$ is an agent then we say that the \textbf{universal intelligence} of $\pi$ is
	\begin{equation} \label{eq:uim}
		\Upsilon(\pi) = \sumop_{\mu \in E}2^{-K(\mu)}V_\mu^\pi
	\end{equation}
	where $V_\mu^\pi$ is the expected reward of the agent in $\mu$,
	\begin{equation} \label{eq:exprewarduim}
		V_\mu^\pi = \mathbb{E}\left(\sumop_{i=1}^\infty r_i\right) \leq 1.
	\end{equation}
\end{definition}

The definition is satisfactory for agents which act synchronously with their environments; that is, the environment waits for the agent to act before giving a new observation. Therefore in Hutter's sense, the framework of \cite{legg2007universal} describes an agent $\pi$ embedded in $\mu$.

Despite this, the environments which we normally consider an intelligent agent to 'act well' in are often chaotic and operate with noise which is temporally independent from the agent-environment interaction itself. \todo{Why!?} For example, a real time game does not wait for a player to press a key, and yet the player still receives perceptual information. The intelligence measure proposed fails to encompass agent-environment interactions where the agent has some delay in acting as the environment continues; modeling such delays as $a_k = \emptyset$ is no more enlightening.

In order to integrate OpenBrain with this framework, we will propose a continuous time universal intelligence measure.

\subsection{Continuous Time Intelligence}

To make a continuous time intelligence measure which is compatible with agents who act instantaneously within an environment, we will define a completed perception space.

Since different agents 

\begin{definition}
	Given an environment $\mu$ with an associated 
	perception space $\mathcal{P}$ we define the
	\textbf{completion} $\tilde{\mathcal{P}}$ with respect to the admissible sequences of observations in $\mu$; that is,
	\begin{equation}
		\tilde{\mathcal{P}} = \bigsqcup_{k=1}^\infty \mathcal{P}^k.
	\end{equation}
\end{definition}


Pre med = jessica $^2q$
\begin{centering}

\begin{tikzpicture}
	\draw[style=dashed, fill=red!30] (2,.5) circle (0.5);
	\draw[fill=green!30] (0,0.5) circle(0.5);
	\path[draw, ->, snake it] (0.5,0.5) -- (4.5,.5) ;
	\path[draw, ->, snake it] (0.5,0.5) -- (4.5,.9) ;
	\path[draw, ->, snake it] (0.5,0.5) -- (4.5,.7) ;
	\path[draw, ->, snake it] (0.5,0.5) -- (4.5,.3) ;
	\path[draw, ->, snake it] (0.5,0.5) -- (4.5,.1) ;
\end{tikzpicture} 

\end{centering}