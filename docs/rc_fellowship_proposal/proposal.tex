%%%%%%%%%%%%%%%%%%%%%%%%%%%%%%%%%%%%%%%%%%%%%%%%%%%%%%%%%%%%%%%%%%
%%%                      Homework _                            %%%
%%%%%%%%%%%%%%%%%%%%%%%%%%%%%%%%%%%%%%%%%%%%%%%%%%%%%%%%%%%%%%%%%%

\documentclass[letter]{article}

\usepackage{lipsum}
\usepackage[pdftex]{graphicx}
\usepackage[margin=1.5in]{geometry}
\usepackage[english]{babel}
\usepackage{listings}
\usepackage{amsthm}
\usepackage{amssymb}
\usepackage{framed}
\usepackage{amsmath}
\usepackage{titling}
\usepackage{fancyhdr}

\pagestyle{fancy}


\newtheorem{theorem}{Theorem}
\newtheorem{definition}{Definition}

\newenvironment{menumerate}{%
  \edef\backupindent{\the\parindent}%
  \enumerate%
  \setlength{\parindent}{\backupindent}%
}{\endenumerate}







%%%%%%%%%%%%%%%
%% DOC INFO %%%
%%%%%%%%%%%%%%%
\newcommand{\bHWN}{_}
\newcommand{\bCLASS}{[DEPT] [CLASSN]}

\title{RC Research Fellowship Proposal: OpenBrain}
\author{William Guss\\26793499\\wguss@berkeley.edu}

\fancyhead[L]{RC Research Fellowship}
\fancyhead[CO]{Proposal}
\fancyhead[CE]{GUSS}
\fancyhead[R]{\thepage}
\fancyfoot[LR]{}
\fancyfoot[C]{}
\usepackage{csquotes}

%%%%%%%%%%%%%%

\begin{document}
\maketitle
\thispagestyle{empty}


%%%%%%% Be sure to set the counter and use menumerate
\section{Background}
Machine learning research in the past decade has had a large focus on variations of the artificial neural network (ANN),
a biologically inspired algorithm originating from the mathematical neuron model proposed by McCulloch and Pitts\cite{mcpitts},
the perceptron neuron proposed by Rosenblatt\cite{perceptron}, and made practical by the back propogation algorithm created by Werbos \cite{bprop}.
Although derivative of biological neural networks, the ANN and its variations stray away from many features involved in biological learning. One
such feature is that learning occurs at the level of the individual biological neuron, whereas the ANN paradigm centralizes this process.
Additionally, ANN learning algorithms don't have a notion of synaptogenesis, which has been shown to be a key component of learning in the biological
brain\cite{gene}. Finally, there is the brain's distribution of processing among billions of individual neurons, something that is difficult to handle
in fully connected paradigms of ANN \cite{annbook}.

\section{Goals}
There are essentially four goals of the OpenBrain project.
\begin{itemize}
\item Build a massiveley parallel Beowulf cluster of parallela computers controlled using MPI on ArchLinux.
\item Create an always online, turing complete modification to the recursive neural network algorithm whose fitness
is determined by the Universal Intelligence Measure described in (Legg and Veness, 2011). The algorithm must have the
following constraints:
    \begin{itemize}
        \item In the spirit of John Conway's turing complete Game of Life, the individual neural nodes must follow
         arbitrarily simple rules in a decentralized fashion.
        \item \emph{Training} is unsupervised and occurs over the lifetime of an \emph{instance} of the open brain,
        such that the aforementioned governing rules are modified with respect to the fitness of the instance.
    \end{itemize}
\item Implement each neural node as an Erlang process distributed across the Beowulf cluster asynchronously.
\item Provide always on input/output to the OpenBrain cluster in similar fashion to that done in Google DeepMind's
  Deep Reinforcement Learning.
\end{itemize}


% \section
\begin{thebibliography}{1}
    \bibitem{annbook} Nicolaos Karayiannis and Anastasios N. Venetsanopoulos {\em
    Artificial Neural Networks: Learning Algorithms, Performance Evaluation, and Applications} 2013: Springer Science
    & Business Media.

    \bibitem{mcpitts} Warren S. McCulloch and Walter Pitts {\em A Logical Calculus of the Ideas Immanent in Nervous Activity} 1943.

    \bibitem{perceptron} Frank Rosenblatt {\em The perceptron: a probabilistic model for information storage and organization in the brain.}
    1958: Psychological review, 65(6), 386.

    \bibitem{bprop} Paul Werbos {\em Beyond Regression: New Tools for Prediction and Analysis in the Behavioral Sciences.} 1974: PhD thesis,
    Harvard University.

    \bibitem{sgene} Monica Hoyos Flight {\em Synaptogenesis: Switching to learn} 2011: Nature Review Neuroscience.

\end{thebibliography}

\end{document}
